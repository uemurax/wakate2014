\PassOptionsToPackage{pdfencoding=auto}{hyperref}
\documentclass{beamer}
\usepackage{luatexja, luatexja-otf}
\usepackage{amsmath, amssymb, amsfonts}
\usepackage{booktabs}
\usepackage{bussproofs}
\title{Univalent Foundations}
\author{Taichi Uemura}
\date{数学基礎論若手の会2014}

\begin{document}

\begin{frame}
  \titlepage
\end{frame}

\begin{frame}
  \frametitle{Outline}
  \tableofcontents
\end{frame}

\section{Introduction}

\begin{frame}
  \frametitle{Homotopy Type Theory}
  refers to a new interpretation of Martin-L\"of's system of
  intensional, constructive type theory into
  abstract homotopy theory.\cite{HoTT}

  \begin{center}
    \begin{tabular}{cc}
      \toprule
      Type & Space \\
      \midrule
      function type
      $\Pi \left( x : A \right) . B \left( x \right)$
      & product space
      $\prod _{x \in A} B \left( x \right)$ \\
      pair type
      $\Sigma \left( x : A \right) . B \left( x \right)$
      & coproduct space
      $\coprod _{x \in A} B \left( x \right)$ \\
      identity type
      $x =_A y$
      & path space
      ${\rm Path}_A\left( x , y \right)$ \\
      \bottomrule
    \end{tabular}
  \end{center}
\end{frame}

\begin{frame}
  \frametitle{Univalent Foundations of Mathematics}
  is Vladimir Voevodsky's new program
  for a comprehensive, computational foundation for mathematics
  based on the homotopical interpretation of type theory.
  \\
  The program is being implemented with proof assistants
  like \href{https://coq.inria.fr/}{Coq}
  or \href{http://wiki.portal.chalmers.se/agda/pmwiki.php}{Agda}.\footnote{
    You can get HoTT libraries at \href{https://github.com/}{GitHub}.
    See \href{https://github.com/HoTT/HoTT}{HoTT/HoTT}
    or \href{https://github.com/HoTT/HoTT-Agda}{HoTT/HoTT-Agda}.
  }
\end{frame}

\begin{frame}
  \frametitle{Key concepts}
  \begin{description}
    \item[Path induction]\mbox{}\\
      powerfull induction principle for path spaces
    \item[Univalent Axiom]\mbox{}\\
      isomorphic types are identical
    \item[Higher Inductive Types]\mbox{}\\
      types with path constructors as well as usual constructors
  \end{description}
\end{frame}

\section{Homotopy Type Theory}

\subsection{Path induction}

\begin{frame}
  \frametitle{Path induction}
  is the induction principle of path spaces as followings:

  \begin{description}
    \item[=-intro]\mbox{}\\
      {$x : A \vdash idp_x : x = x$}
    \item[=-elim]\mbox{}\\
      \AxiomC{$x \ y : A , p : x = y \vdash D \left( p \right) : Type$}
      \AxiomC{$x : A \vdash d : D \left( idp_x \right)$}
      \BinaryInfC{$x \ y : A , p : x = y \vdash ind \left( d , p \right) : D \left( p \right)$}
      \DisplayProof
    \item[=-comp]\mbox{}\\
      {$x : A \vdash ind \left( d , idp_x \right) \equiv d$}
  \end{description}

  i.e.
  \begin{itemize}
    \item to show that for all $x$ and $y$ such that $x = y$, $P \left( x , y \right)$,
      it is enough to show that
      for all $x$, $P \left( x , x \right)$.
    \item $\left( - \right) = \left( - \right)$ is
      an inductive type family generated by identity paths.
  \end{itemize}
\end{frame}

\begin{frame}[containsverbatim]
  \frametitle{Examples}
  We can show that any type forms a {\it groupoid}\footnote{
    in fact an {\it \omega-groupoid}.}.
  \\
  For example, using Agda:

  \begin{verbatim}
!_ : {x y : A} (p : x == y) -> y == x
! idp = idp

_*_ : {x y z : A} (p : x == y) (q : y == z) -> x == z
p * idp = p

*-assoc : {x y z w : A}
  (p : x == y) (q : y == z) (r : z == w)
  -> p * (q * r) == (p * q) * r
*-assoc p q idp = idp
  \end{verbatim}
\end{frame}

\subsection{Univalent Axiom}

\begin{frame}
  \frametitle{Homotopies and Equivalences}
  We can define the notions of {\it homotopies} and {\it equivalences}
  in natural ways:
  \begin{itemize}
    \item
    $f \sim g :\equiv \Pi \left( x : A \right) .
    f \left( x \right) = g \left( x \right)$
    \item
    $A \simeq B :\equiv \Sigma \left( f : A \to B \right) .
    \Sigma \left( g : B \to A \right) .$ \\
    $\Sigma \left( \epsilon : f \circ g \sim id \right) .
    \Sigma \left( \eta : g \circ f \sim id \right) .$ \\
    $\Pi \left( x : A \right) .
    f \left( \eta \left( x \right) \right) = \epsilon \left( f \left( x \right) \right)$
  \end{itemize}
\end{frame}

\begin{frame}
  \frametitle{Univalent Axiom}
  postulates
  $$
  ua : \Pi \left( A \ B : Type \right) .
  \left( A \simeq B \right) \simeq \left( A = B \right)
  $$
  Using path induction,
  we often assume that all equivalences are identity maps.
\end{frame}

\subsection{Higher Inductive Types}

\begin{frame}
  \frametitle{Higher Inductive Types}
  are inductive types with {\it path constructors}. \\
  For example, the unit circle ${\mathbb S}^1$ is generated by:
  \begin{itemize}
    \item a point constructor $base : {\mathbb S}^1$
    \item a path constructor $loop : base = base$
  \end{itemize}
  ${\mathbb S}^1$ has a {\it recursion principle}\footnote{
    ${\mathbb S}^1$ has also an {\it induction principle},
    but it is a bit complicated.}:
  $$
  \left( {\mathbb S}^1 \to A \right) \simeq
  \left( \Sigma \left( x : A \right) . x = x \right)
  $$
  In other words, ${\mathbb S}^1$ is the {\it free \omega-groupoid}
  generated by
  \begin{itemize}
    \item a 0-cell $base : {\mathbb S}^1$
    \item a 1-cell $loop : base = base$
  \end{itemize}
\end{frame}

\begin{frame}
  \frametitle{Example}
  Combined with the Univalent Axiom,
  we can construct the {\it universal cover} of ${\mathbb S}^1$:
  $$
    \widetilde{ {\mathbb S}^1 } : {\mathbb S}^1 \to Type
  $$
  $$
    \widetilde{ {\mathbb S}^1 } \left( base \right) :\equiv {\mathbb Z}
  $$
  $$
    \widetilde{ {\mathbb S}^1 } \left( loop \right) := ua \left( succ \right)
  $$
  where ${\mathbb Z}$ is the type of integers
  and $succ : {\mathbb Z} \simeq {\mathbb Z}$ is the successor.

  Using this construction, we can show that
  the {\it fundamental group} of ${\mathbb S}^1$ is ${\mathbb Z}$:
  $$
  \left( base = base \right) \simeq {\mathbb Z}
  $$
\end{frame}

\section{References}

\begin{thebibliography}{99}
\bibitem{HoTT}[HoTT]
  Homotopy Type Theory
  \url{http://homotopytypetheory.org/}
\end{thebibliography}

\end{document}


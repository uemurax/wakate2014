\documentclass[dvipdfmx]{beamer}
\usepackage{etex}
\usepackage{amsmath, amssymb, amsfonts, amsthm}
\usepackage{array}
\usepackage{bussproofs}
\usepackage[all]{xy}
\title{Univalent Foundations}
\author{Taichi Uemura}
\institute[KURIMS]{\url{uemura@kurims.kyoto-u.ac.jp}
    \\
    Research Institute for Mathematical Science,
    Kyoto University
  }
\date{November 19, 2014}

\begin{document}

\section{Top}

\begin{frame}
  \titlepage
  \footnotetext{\url{https://github.com/uemurax/wakate2014}}
\end{frame}

\begin{frame}
  \frametitle{Outline}
  \tableofcontents
\end{frame}

\section{Introduction}

\begin{frame}
  \frametitle{Homotopy Type Theory}
  refers to a new interpretation of Martin-L\"of's system of
  intensional, constructive type theory into
  abstract homotopy theory.\cite{hottbook}

  \begin{center}
    \begin{tabular}{c|c}
      \hline
      Type & Space \\
      \hline
      function type
      $\Pi \left( x : A \right) . B \left( x \right)$
      & product space
      $\prod _{x \in A} B \left( x \right)$ \\
      pair type
      $\Sigma \left( x : A \right) . B \left( x \right)$
      & coproduct space
      $\coprod _{x \in A} B \left( x \right)$ \\
      identity type
      $x =_Amidrule y$
      & path space
      ${\rm Path}_A\left( x , y \right)$ \\
      \hline
    \end{tabular}
  \end{center}
\end{frame}

\begin{frame}
  \frametitle{Univalent Foundations of Mathematics}
  is Vladimir Voevodsky's new program
  for a comprehensive, computational foundation for mathematics
  based on the homotopical interpretation of type theory.
  \\
  The program is being implemented with proof assistants
  like \href{https://coq.inria.fr/}{Coq}
  or \href{http://wiki.portal.chalmers.se/agda/pmwiki.php}{Agda}.\footnote{
    You can get HoTT libraries at \href{https://github.com/}{GitHub}.
    See \href{https://github.com/HoTT/HoTT}{HoTT/HoTT}
    or \href{https://github.com/HoTT/HoTT-Agda}{HoTT/HoTT-Agda}.
  }
\end{frame}

\begin{frame}
  \frametitle{Key concepts}
  \begin{description}
    \item[Path induction]\mbox{}\\
      powerfull induction principle for path spaces
    \item[Univalence Axiom]\mbox{}\\
      isomorphic types are identical
    \item[Higher Inductive Types]\mbox{}\\
      types with path constructors as well as usual constructors
  \end{description}
\end{frame}

\section{Homotopy Type Theory}

\subsection{Path induction}

\begin{frame}
  \frametitle{Path induction}
  is the induction principle of path spaces as followings:

  \begin{description}
    \item[=-intro]\mbox{}\\
      {$x : A \vdash idp_x : x = x$}
    \item[=-elim]\mbox{}\\
      \AxiomC{$x \ y : A , p : x = y \vdash D \left( p \right) : Type$}
      \AxiomC{$x : A \vdash d : D \left( idp_x \right)$}
      \BinaryInfC{$x \ y : A , p : x = y \vdash
      ind_{=} \left( d , p \right) : D \left( p \right)$}
      \DisplayProof
    \item[=-comp]\mbox{}\\
      {$x : A \vdash ind_{=} \left( d , idp_x \right) \equiv d$}
  \end{description}

  i.e.
  \begin{itemize}
    \item to show that for all $x$ and $y$ such that $x = y$, $P \left( x , y \right)$,
      it is enough to show that
      for all $x$, $P \left( x , x \right)$.
    \item $\left( - \right) = \left( - \right)$ is
      an inductive type family generated by identity paths.
  \end{itemize}
\end{frame}

\begin{frame}[containsverbatim]
  \frametitle{Examples - groupoid structure of path spaces}
  We can show that any type forms a {\it groupoid}.\footnote{
    in fact an {\it $\omega$-groupoid}}
  \\
  For example, using Agda:

  \begin{verbatim}
!_ : {x y : A} (p : x == y) -> y == x
! idp = idp

_*_ : {x y z : A} (p : x == y) (q : y == z) -> x == z
p * idp = p

*-assoc : {x y z w : A}
  (p : x == y) (q : y == z) (r : z == w)
  -> p * (q * r) == (p * q) * r
*-assoc p q idp = idp
  \end{verbatim}
\end{frame}

\begin{frame}[containsverbatim]
  \frametitle{Examples - functions are functors}
  \begin{verbatim}
ap : (f : A) {x y : A} (p : x == y) -> f x == f y
ap f idp = idp
  \end{verbatim}
  We often write simply
  $f \left( p \right)$
  for $ap_f \left( p \right)$.
\end{frame}

\subsection{Univalence Axiom}

\begin{frame}
  \frametitle{Homotopies and Equivalences}
  We can define the notions of {\it homotopies} and {\it equivalences}
  in natural ways:
  \begin{itemize}
    \item
    $f \sim g :\equiv \Pi \left( x : A \right) .
    f \left( x \right) = g \left( x \right)$
    \item
    $A \simeq B :\equiv \Sigma \left( f : A \to B \right) .
    \Sigma \left( g : B \to A \right) .$ \\
    $\Sigma \left( \epsilon : f \circ g \sim id \right) .
    \Sigma \left( \eta : g \circ f \sim id \right) .$ \\
    $\Pi \left( x : A \right) .
    f \left( \eta \left( x \right) \right) = \epsilon \left( f \left( x \right) \right)$
  \end{itemize}
  This equivalence is so called {\it ``half adjoint equivalence''}
  in higher category theory.
\end{frame}

\begin{frame}
  \frametitle{Univalence Axiom}
  postulates
  \[
    ua : \Pi \left( A \ B : Type \right) .
    \left( A \simeq B \right) \simeq \left( A = B \right)
  \]
  Using path induction,
  we often assume that all equivalences are identity maps:

  \begin{theorem}[Equivalence induction]
    Given a type family
    $D : \Pi \left( A , B : Type \right) . \left( A \simeq B \right)
    \to Type$ and a function
    $d : \Pi \left( A : Type \right) . D \left( A , A , id_A \right)$,
    there exists
    $f : \Pi \left( A , B : Type \right) .
    \Pi \left( e : A \simeq B \right) .
    D \left( A , B , e \right)$ such that
    for all $A : Type$,
    $f \left( A , A , id_A \right) = d \left( A \right)$.
  \end{theorem}
\end{frame}

\begin{frame}
  \frametitle{Example - type formers are functors}
  \framesubtitle{(w.r.t. equivalences)}
  We will be able to show that
  individual type formers ($\Pi$, $\Sigma$, \dots) are functors.
  \\
  But how about {\it all} type formers?
  \\
  The Univalence Axiom forces them to be functors w.r.t. equivalences:
  \\
  Let $P : Type \to Type$.
  Then we have
  \[
    ap_P : \Pi \left( A , B : Type \right) .
    \left( A \simeq B \right) \to
    \left( P \left( A \right) \simeq P \left( B \right) \right)
  \]
  \[
    ap_P \left( id \right) = id
  \]
  by equivalence induction.
\end{frame}

\subsection{Higher Inductive Types}

\begin{frame}
  \frametitle{Higher Inductive Types}
  (HITs) are inductive types with {\it path constructors}. \\
  For example, the unit circle ${\mathbb S}^1$ is generated by:
  \begin{itemize}
    \item a point constructor $base : {\mathbb S}^1$
    \item a path constructor $loop : base = base$
  \end{itemize}
  ${\mathbb S}^1$ has a {\it recursion principle}\footnote{
    ${\mathbb S}^1$ has also an {\it induction principle},
    but it is a bit complicated.}:
  \[
    \left( {\mathbb S}^1 \to A \right) \simeq
    \left( \Sigma \left( x : A \right) . x = x \right)
  \]
  In other words, ${\mathbb S}^1$ is the {\it free $\omega$-groupoid}
  generated by
  \begin{itemize}
    \item a 0-cell $base : {\mathbb S}^1$
    \item a 1-cell $loop : base = base$
  \end{itemize}
\end{frame}

\begin{frame}
  \frametitle{Example - the universal cover of ${\mathbb S}^1$}
  Combined with the Univalence Axiom,
  we can construct the {\it universal cover} of ${\mathbb S}^1$:
  \[
    \widetilde{ {\mathbb S}^1 } : {\mathbb S}^1 \to Type
  \]
  \[
    \widetilde{ {\mathbb S}^1 } \left( base \right) :\equiv {\mathbb Z}
  \]
  \[
    \widetilde{ {\mathbb S}^1 } \left( loop \right) := ua \left( succ \right)
  \]
  where ${\mathbb Z}$ is the type of integers
  and $succ : {\mathbb Z} \simeq {\mathbb Z}$ is the successor.

  Using this construction, we can show that
  the {\it fundamental group} of ${\mathbb S}^1$ is ${\mathbb Z}$:
  \[
    \left( base = base \right) \simeq {\mathbb Z}
  \]
\end{frame}

\begin{frame}
  \frametitle{Another example - free groups}
  In classical type theory,
  we can construct the {\it free monoid} on a type $A$,
  namely the type $List \left( A \right)$ of finite lists
  of elements of A.
  \\
  However, cannot the {\it free group} on $A$.
  \\
  Higher inductive types make it possible:
  \\
  Define a type $F \left( A \right)$ is generated by
  \begin{itemize}
    \item a function $\eta : A \to F \left( A \right)$
    \item a function $m : F \left( A \right) \to
      F \left( A \right) \to F \left( A \right)$
    \item an element $e : F \left( A \right)$
    \item a function $i : F \left( A \right) \to F \left( A \right)$
    \item an inhabitant of
      {\it ``$\left( F \left( A \right) , m , e , i \right)$ is a group''}
  \end{itemize}
  Notice that the statement ``$G$ is a group''
  is written in equations, so that
  $F \left( A \right)$ is defined as a higher inductive type.
\end{frame}

\begin{frame}
  \frametitle{Universal property of free group}
  As for ${\mathbb S}^1$,
  the free group $F \left( A \right)$ on $A$
  is characterized by a recursion principle:
  \\
  For any group $G$,
  \[
    Hom_{Grp} \left( F \left( A \right) , G \right)
    \simeq
    \left( A \to G \right)
  \]
  This represents the {\it universal property}
  of the free group.
\end{frame}

\section{Models of HoTT}

\begin{frame}
  \frametitle{How to interpret the type theory?}
  We want a {\it model} of the Homotopy Type Theory
  to show that the theory is consistent relative to,
  for instance, ZFC
  \\
  or
  to obtain a {\it computational} interpretation of
  the Univalence Axiom.\nocite{bezem_et_all:LIPIcs:20144628}
\end{frame}

%\begin{frame}
  %\frametitle{A C-system\footnote{
    %originally, C-systems were introduced by Cartmell
    %\cite{Cartmell1986209}
    %under the name ``contextual categories''.
    %The name ``C-system'' is given by Voevodsky
    %\cite{voevodsky2014subsystems}.
    %}
  %}
  %is a category $CC$ which has the notion of
  %{\it ``type dependencies''}:
  %\begin{itemize}
    %\item $CC$ has a final object $pt$
    %\item all $\Gamma \in Ob \left( CC \right)$ but $pt$
      %have canonical projections
      %$p_{\Gamma} : \Gamma \to ft \left( \Gamma \right)$
    %\item canonical projections can be pulled back
      %with strict functoriality
  %\end{itemize}
  %so that every $\Gamma \in Ob \left( CC \right)$
  %has the chain
  %\[
  %\Gamma \overset{p}{\to} ft \left( \Gamma \right)
  %\overset{p}{\to} ft^2 \left( \Gamma \right)
  %\overset{p}{\to} \dots
  %\overset{p}{\to} pt
  %\]
  %of dependencies.
%\end{frame}

\subsection{Factorization systems}

\begin{frame}
  \frametitle{A weak factorization system
  \nocite{warren2008homotopy,awodey2007homotopy}}
  consists of:
  \begin{itemize}
    \item two collections ${\mathcal L}$ ({\it acyclic cofibrations})
      and ${\mathcal R}$ ({\it fibrations})
      of maps in a category ${\mathbb C}$
      such that
      \begin{itemize}
        \item every map $f : A \to B$ has a factorization as
          \[
            \xymatrix{
              A \ar[rr]^{i} \ar[dr]_{f}
              &
              & C \ar[dl]^{p}
              \\
              & B
              &
            }
          \]
          where $i \in {\mathcal L}$ and $p \in {\mathcal R}$
        \item a map $f : A \to B$ is an acyclic cofibration iff
          $f$ has the {\it left lifting property} w.r.t.
          all fibrations
        \item a map $g : C \to D$ is a fibration iff
          $g$ has the {\it right lifting property} w.r.t.
          all acyclic cofibrations
      \end{itemize}
  \end{itemize}
  \[
    \xymatrix{
      A \ar[r]^{h} \ar[d]_{ {\mathcal L} \ni f}
      & C \ar[d]^{g \in {\mathcal R}}
      \\
      B \ar[r]_{k} \ar@{.>}[ur]^{J}
      & D
    }
  \]
\end{frame}

\begin{frame}
  \frametitle{Model categories}
  Weak factorization systems are often provided
  by {\it model structures}.
  \nocite{Quillen:1967,hovey2007model}
  \\
  For example
  \begin{itemize}
    \item the category ${\rm Top}$ of topological spaces
      with {\it Serre fibrations}
    \item the category ${\rm SSet}$ of simplicial sets
      with {\it Kan fibrations}
    \item the category ${\rm Gpd}$ of groupoids
      with {\it Grothendieck fibrations}
  \end{itemize}
  these are arose in homotopy theory or
  higher category theory.
  \\
  Kapulkin, Lumsdaine and Voevodsky showed that
  ${\rm SSet}$ satisfies the Univalence Axiom.
  \cite{kapulkin2012simplicial,kapulkin2012univalence}
\end{frame}

\begin{frame}
  \frametitle{The interpretation of type families}
  While the idea of the Curry-Howard correspondence is
  ``Propositions as Types'',
  that of the homotopical interpretation is
  \begin{center}
    {\it Fibrations as Types}.
  \end{center}
  Let $\left( {\mathcal L} , {\mathcal R} \right)$
  be a weak factorization system in a category ${\mathbb C}$.
  \\
  A type family $x : A \vdash B \left( x \right) : Type$
  will be interpreted as a {\it fibration} $p_{B} : B \to A$
  and an inhabitant
  $x : A \vdash b \left( x \right) : B \left( x \right)$
  as a {\it section} of
  $B \to A$:
  \[
    \xymatrix{
      & B \ar[d]^{p_{B}}
      \\
      A \ar@{=}[r] \ar[ur]^{b}
      & A
    }
  \]
\end{frame}

\begin{frame}
  \frametitle{The interpretation of identity types}
  For every fibration $A \to \Gamma$
  we have a factorization of a diagonal map
  \[
    \xymatrix{
      A \ar[rr]^{r} \ar[dr]_{\Delta}
      &
      & Id_{A} \ar[dl]^{p_{Id_A}}
      \\
      & A \times _{\Gamma}  A
      &
    }
  \]
  over $\Gamma$, where $r$ is an acyclic cofibration
  and $p_{Id_A}$ is a fibration.
  \\
  Using lifting property,
  the induction principle can be interpreted as
  \[
    \xymatrix{
      A \ar[rr]^{d} \ar[dd]_{r}
      &
      & D \ar[dd]^{p_{D}}
      \\
      & & \\
      Id_{A} \ar@{=}[rr] \ar@{.>}[urur]^{ind_{=} \left( d \right)}
      &
      & Id_{A}
    }
  \]
\end{frame}

%\begin{frame}
  %\frametitle{Universe in a category ${\mathbb C}$}
  %is an morphism $p : \tilde{U} \to U$ in ${\mathbb C}$
  %together with a mapping which assigns to any morphism
  %$A : \Gamma \to U$ in ${\mathbb C}$ a pull-back square
  %\[
    %\xymatrix{
      %\left( \Gamma , A \right) \ar[r]^{Q \left( A \right)}
      %\ar[d]_{p \left( \Gamma , A \right)}
        %& \tilde{U} \ar[d]^{p}
        %\\
        %\Gamma \ar[r]^{A}
        %& U
    %}
  %\]
  %%We think of $A$ as a {\it characteristic function} of
  %%$\left( \Gamma , A \right)$.
  %We think of $U$ as a set of types and
  %$\tilde{U}$ as a set of terms.
%\end{frame}
%
%\begin{frame}
  %\frametitle{Interpretation of identity types}
  %{\it Id-structure} on a universe consists of:
  %\begin{itemize}
    %\item $Id : \tilde{U} \times _{U} \tilde{U} \to U$
    %and $r : \tilde{U} \to
    %\left( \tilde{U} \times _{U} \tilde{U} , Id\right)$
    %such that the triangle
    %\[
      %\xymatrix{
        %\tilde{U} \ar[rr]^{r}
        %\ar[dr]_{\Delta _{\tilde{U}}}
        %&
        %& \left( \tilde{U} \times _{U} \tilde{U} , Id \right)
        %\ar[dl]^{p \left( \tilde{U} \times _{U} \tilde{U} , Id \right)}
        %\\
        %& \tilde{U} \times _{U} \tilde{U}
        %&
      %}
    %\]
    %commutes
    %\item the following {\it lifting property}
      %in ${\mathbb C} / U$:
      %\[
        %\xymatrix{
          %\tilde{U} \ar[r]^{d} \ar[d]_{r}
          %& \tilde{U} \times U \ar[d]^{p \times U}
          %\\
          %\left( \tilde{U} \times _{U} \tilde{U} , Id \right)
          %\ar@{.>}[ur]^{ind_{=}} \ar[r]_{D}
          %& U \times U
        %}
      %\]
  %\end{itemize}
%\end{frame}

%\subsection{Simplicial sets}
%
%\begin{frame}
  %\frametitle{Models in ${\rm SSet}$}
  %Kapulkin, Lumsdaine and Voevodsky constructed
  %a model of the Univalent Foundations of Mathematics
  %in the category ${\rm SSet}$ of {\it simplicial sets} and
  %showed that in this model
  %the Univalence Axiom holds
  %\cite{kapulkin2012simplicial,kapulkin2012univalence}.
%\end{frame}
%
%\begin{frame}
  %\frametitle{Simplicial sets}
  %are generalizations of the geometric simplicial complexes.
  %\nocite{friedman2008elementary}
  %A simplicial set is built up out of gluing abstract simpleces
  %to each other.
%\end{frame}

\section{Next stage}

\begin{frame}
  \frametitle{Related topics and future works}
  \begin{itemize}
    \item about HITs
      \begin{itemize}
        \item general syntax of HITs?
        \item models with HITs?
      \end{itemize}
    \item {\it computational} interpretation of
      univalence and HITs
      \begin{itemize}
        \item {\it cubical sets}
          \cite{bezem_et_al:LIPIcs:2014:4628}
      \end{itemize}
    \item general semantics of dependent type theories
      \begin{itemize}
        \item contextual categories
          \cite{Cartmell1986209}
        \item C-systems and B-systems
          \cite{voevodsky2014subsystems,voevodsky2014bsystems}
      \end{itemize}
    \item homotopy theory and algebraic topology
      \begin{itemize}
        \item synthetic homotopy theory
        \item spaces as HITs, for example
          torus, projective plane, Klein bottle, $\dots$
        \item homotopy groups of spheres
      \end{itemize}
    \item and more $\dots$
  \end{itemize}
\end{frame}

%\subsection{Cubical sets}
%
%\begin{frame}
  %\frametitle{Cubical sets}
  %Bezem, Coquand and Huber constructed
  %a model of dependent type theory
  %in the category of cubical sets.
  %\cite{bezem_et_al:LIPIcs:2014:4628}
%\end{frame}

\section{References}

\bibliographystyle{plain}
\bibliography{UF-slide}

\end{document}

